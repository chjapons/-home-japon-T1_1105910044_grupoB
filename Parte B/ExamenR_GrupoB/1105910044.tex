\documentclass[12pt]{report}
\usepackage[spanish]{babel}
\usepackage[utf8]{inputenc}
\usepackage{amsmath,amsthm,amsfonts,amssymb}
\usepackage{Sweave}
\usepackage{graphicx}
\usepackage{hyperref}
\usepackage{anysize} 
\marginsize{1.42cm}{1.36cm}{1.48cm}{1.48cm} 

\title{\includegraphics[width=7cm, height=7cm]{unl.jpg}}

\author{\\\\ Carlos Humberto Japón Suquilanda \\\\ ECINF7223 \\\\ \texttt{chjapons@unl.edu.ec}}

\begin{document}

\maketitle

\textbf{Contestar la siguiente interrogante:}\\

\textbf{¿Es aplicable la ingeniería de software cuando se elaboran webapps?}\\

Si es aplicable.\\

\textbf{Si es así, ¿cómo puede modificarse para que asimile las características únicas de éstas?}\\

Para que asimilen las caracteristicas es necesario tomar en cuenta parametros para el mejoramiento de la webapps.\\

\textbf{Rendimiento}\\

El rendimiendo es esencial para que un usuario de la webapp no deba esperar demasiado para utilizar sus servicios y no decida irse a otra parte.\\

\textbf{Orientadas a los datos.}\\

La función principal de muchas webapp es el uso de hipermedios para presentar al usuario final contenido en forma de texto, gráficas, audio y video. Además, las webapps se utilizan en forma común para acceder a información que existe en bases de datos que no son parte integral del ambiente basado en web (por ejemplo, comercio electrónico o aplicaciones financieras).\\

\textbf{Evolución continua.}\\

A diferencia del software de aplicación convencional que evoluciona a lo largo de una serie de etapas planeadas y separadas cronológicamente, las aplicaciones
web evolucionan en forma continua.\\

No es raro que ciertas webapp (específicamente su contenido) se actualicen minuto a minuto o que su contenido se calcule en cada solicitud.\\

\textbf{Concurrencia}\\

A la webapp puede acceder un gran número de usuarios a la vez. En muchos casos, los patrones de uso entre los usuarios finales varían mucho.\\

\textbf{Disponibilidad}\\

hasta ahora no es posible sperar una disponibilidad de 100%, es frecuente
que los usuarios de webapps populares demanden acceso las 24 horas de los 365 días del año.\\

\textbf{Contenido sensible}\\

La calidad y naturaleza estética del contenido constituye un rasgo importante de la calidad de una webapp.\\


\textbf{Descripción del dataset Titanic}\\
\begin{center}

\textbf{La supervivencia de los pasajeros del Titanic}\end{center}

\textbf{Descripción}\\

Este conjunto de datos proporciona información sobre el destino de los pasajeros en el primer viaje fatal del trasatlántico Titanic, que se resume de acuerdo a la situación económica (clase), el sexo, la edad y la supervivencia.\\

\textbf{Uso}\\

Titanic.\\

\textbf{Formato}\\

Una matriz de cuatro dimensiones resultante de la cruzada tabular de 2201 observaciones sobre 4 variables. Las variables y sus niveles son los siguientes:\\

Nombres de	Niveles\\
1	Clase	Primero, segundo, tercero, Tripulación\\
2	Sexo	Hombre, Mujer\\
3	Años	Niño, Adulto\\
4	Sobrevivieron	No si\\

\textbf{Detalles}\\

El hundimiento del Titanic es un evento famoso, y nuevos libros siguen siendo publicado sobre el tema. Muchos hechos conocidos de las proporciones de los pasajeros de primera clase a la política de "mujeres y niños primero ', y el hecho de que esa política no era un éxito completo en el caso de las mujeres y niños en la tercera clase, se reflejan en la supervivecia de diversas clases de pasajeros seguin su tarifa.\\

Estos datos fueron recogidos originalmente por la Junta Británica de Comercio en su investigación del hundimiento. Tenga en cuenta que no hay un acuerdo completo entre las fuentes primarias como a las cifras exactas a bordo, rescatados, o perdidos.\\

Debido, en particular, a la película de gran éxito 'Titanic', los últimos años vieron un aumento en el interés público en el Titanic. Datos muy detallados sobre los pasajeros ya está disponible en Internet, en sitios como la Enciclopedia Titanica (http://www.rmplc.co.uk/eduweb/sites/phind).\\

\textbf{Fuente}\\

Dawson, Robert J. MACG. (1995), El 'Episodio inusual' Datos Revisited. Diario de Estadísticas de Educación, 3. Http://www.amstat.org/publications/jse/v3n3/datasets.dawson.html

La fuente proporciona un conjunto de datos de clase de grabación, el sexo, la edad y el estado de supervivencia para cada persona a bordo del Titanic, y se basa en datos recogidos originalmente por la Junta Británica de Comercio y reimpresos en:

Junta Británica de Comercio (1990), Informe sobre la pérdida del 'Titanic' (SS). Junta Británica de Comercio Informe Investigación (reimpresión). Gloucester, Reino Unido: Allan Sutton Publishing.\\

\textbf{Mostrarel Dataset Titanic}\\

\textbf{Tabla:}\\
\begin{Schunk}
\begin{Soutput}
, , Age = Child, Survived = No

      Sex
Class  Male Female
  1st     0      0
  2nd     0      0
  3rd    35     17
  Crew    0      0

, , Age = Adult, Survived = No

      Sex
Class  Male Female
  1st   118      4
  2nd   154     13
  3rd   387     89
  Crew  670      3

, , Age = Child, Survived = Yes

      Sex
Class  Male Female
  1st     5      1
  2nd    11     13
  3rd    13     14
  Crew    0      0

, , Age = Adult, Survived = Yes

      Sex
Class  Male Female
  1st    57    140
  2nd    14     80
  3rd    75     76
  Crew  192     20
\end{Soutput}
\end{Schunk}

\textbf{Cuál es el numero total de casos en el dataset}\\\\

\textbf{El número total de casos del daset Titanic es:}\\

\begin{Schunk}
\begin{Soutput}
[1] 2201
\end{Soutput}
\end{Schunk}

\begin{center}
\includegraphics[width=6cm, height=1.5cm]{cc.jpg}
\end{center}
\end{document}
